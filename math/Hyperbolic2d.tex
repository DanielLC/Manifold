\documentclass[12pt]{amsart}
\usepackage{a4}
\usepackage{amsmath,amssymb,amsthm}
\usepackage{multicol}
\usepackage{verbatim}
\newcommand{\sgn}{\mathop{\mathrm{sgn}}}
\newcommand{\ignore}[1]{}
\newcommand{\mat}[4]{\left(\begin{array}{ccc} #1 & #2 \\#3 & #4 \end{array} \right)}
\newcommand{\vect}[2]{\left(\begin{array}{ccc} #1 \\#2 \end{array} \right)}
\newcommand{\matc}[9]{\left(\begin{array}{ccc} #1 & #2 & #3 \\#4 & #5 & #6 \\#7 & #8 & #9 \end{array} \right)}
\begin{document}

Finding the direction and distance from one point to another in $H^2$:

We will be using the half plane model: $\textbf{H}^2 \mapsto \{(x_1,x_2):x_2 > 0\}$.

Given points $\textbf{x} = (x_1,x_2)$ and $\textbf{y} = (y_1,y_2)$,

The geodesic between them is mapped to a circle on the half-plane with a center on the $x$-axis. Consider this circle.

Let $\textbf{c}$ be the center of the circle, and $r$ be the radius. We know $c_2 = 0, \|\textbf{x}-\textbf{c}\| = \|\textbf{y}-\textbf{c}\| = r$.

First, let us solve for $c_1$.

$(x_1-c_1)^2+{x_2}^2 = (y_1-c_1)^2+{y_2}^2 = r^2$

$= {x_1}^2-2x_1c_1+c_1^2+{x_2}^2 = {y_1}^2-2y_1c_1+c_1^2+{y_2}^2$

${x_1}^2-2x_1c_1+{x_2}^2 = {y_1}^2-2y_1c_1+{y_2}^2$

$2(x_1-y_1)c_1 = {x_1}^2+{x'_2}^2-{y_1}^2-{y_2}^2$

$c_1 = \frac{({x_1}^2+{x_2}^2)-({y_1}^2+{y_2}^2)}{2(x_1-y_1)}$

$= \frac{\|\textbf{x}\|^2-\|\textbf{y}\|^2}{2(x_1-y_1)}$

We can use this value of $c_1$ to compute $r$.

The vector at $\textbf{x}$ pointing to $\textbf{y}$ is tangent the circle, which means that its a right angle from the direction to $\textbf{c}$, which works out to be $\mat{0}{-1}{1}{0}(\textbf{c}-\textbf{x}) = \vect{x_2}{c_1-x_1}$. We then normalize this to $\frac{(x_2,c_1-x_1)}{\sqrt{x_2^2+(c_1-x_1)^2}}$.

The distance along the geodesic can be calculated using the angle of the arc between $\textbf{x}$ and $\textbf{y}$ on the half plane. Assuming it moves counterclockwise:

$\int_{\theta_1}^{\theta_2} \frac{\sqrt{(\frac{d}{d\theta}r\sin\theta)^2+(\frac{d}{d\theta}r\cos\theta)^2}}{r\sin\theta} d\theta$

$= \int_{\theta_1}^{\theta_2} \frac{\sqrt{(r\cos\theta)^2+(-r\sin\theta)^2}}{r\sin\theta} d\theta$

$= \int_{\theta_1}^{\theta_2} \frac{r}{r\sin\theta} d\theta$

$= \int_{\theta_1}^{\theta_2} \csc\theta d\theta$

$= -\ln\left|\frac{\csc\theta_2+\cot\theta_2}{\csc\theta_1+\cot\theta_1}\right|$

$= \ln\left|\frac{\csc\theta_1+\cot\theta_1}{\csc\theta_2+\cot\theta_2}\right|$

$\csc\theta_1 = \frac{r}{x_2}, \cot\theta_1 = \frac{x_1-c_1}{x_2}, \csc\theta_2 = \frac{r}{y_2}, \cot\theta_2 = \frac{y_1-c_1}{y_2}$

Hence, the distance is $\ln\left|\frac{\frac{r}{x_2}+\frac{x_1-c_1}{x_2}}{\frac{r}{y_2}+\frac{y_1-c_1}{y_2}}\right|$

$= \ln\left|\frac{\frac{r+x_1-c_1}{x_2}}{\frac{r+y_1-c_1}{y_2}}\right|$

$= \ln\left|\frac{y_2(r+x_1-c_1)}{x_2(r+y_1-c_1)}\right|$

Since $\|\textbf{x}-\textbf{c}\|, \textbf{y}-\textbf{c}\| > r$, clearly $|x_1-c_1|, |y_1-c_1| > r$, and we already know $x_2$ and $y_2 > 0$, so $\frac{y_2(r+x_1-c_1)}{x_2(r+y_1-c_1)} \geq 0$. Thus, we may remove the absolute value.

$= \ln\frac{y_2(r+x_1-c_1)}{x_2(r+y_1-c_1)}$

If the path is clockwise instead of counterclockwise, it comes out to $-\ln\frac{y_2(r+x_1-c_1)}{x_2(r+y_1-c_1)} = \ln\frac{x_2(r+y_1-c_1)}{y_2(r+x_1-c_1)}$. In general, the distance is $\left|\ln\frac{y_2(r+x_1-c_1)}{x_2(r+y_1-c_1)}\right|$.

Of course, none of this works if the two points share the same first two coordinates. In that case, the problem is easier. The direction is $(0,1)$, and the distance is $\int_{x_3}^{y_3} \frac{1}{t} dt = \ln\frac{y_3}{x_3}$. This gives a vector of $(0,\ln\frac{y_3}{x_3})$.

\bigskip

Finding the point a given distance in a given direction from another:

\bigskip

Given initial point $\textbf{x} = (x_1,x_2)$ and vector $\textbf{z} = (z_1,z_2)$,

Since the geodesic is a circle, the center of the circle is on a line perpendicular to the tangent vector.

The tangent line is $\textbf{x}+t\textbf{z}$, so the center of the circle is on $\textbf{x}+t\mat{0}{-1}{1}{0}\textbf{z} = (x_1-tz_2,x_2+tz_1)$.

This intersects with the $x$ axis when $x_2+tz_1 = 0$.

$t = -\frac{x_2}{z_1}$

$c_1 = x_1-tz_2$

$= x_1+\frac{x_2z_2}{z_1}$

%In the special case of $z_1 = 0$

And as before, $c_2 = 0$.

$r = \|\textbf{x}-\textbf{c}\|$

If were letting $x_1 = 0$, that's just $r^2 = c_1^2+{x_2}^2$.

Now that we have the circle, its just a matter of finding the point at the right distance.

Let $d = \ln\frac{y_2(r+x_1-c_1)}{x_2(r+y_1-c_1)}$

$e^d = \frac{y_2(r+x_1-c_1)}{x_2(r+y_1-c_1)}$

$x_2(r+y_1-c_1)e^d = y_2(r+x_1-c_1)$

${x_2}^2(r+y_1-c_1)^2e^{2d} = {y_2}^2(r+x_1-c_1)^2$

$= [r^2-(y_1-c_1)^2](r+x_1-c_1)^2$

$= [r-(y_1-c_1)][r+(y_1-c_1)](r+x_1-c_1)^2$

$= (r-y_1+c_1)(r+y_1-c_1)(r+x_1-c_1)^2$

Since $y_0 > 0$ and $r^2 = |\textbf{y}-\textbf{c}|$, it must be the case that $|y_1-c_1| < r$ so $r+y_1-c_1 \neq 0$ and can be safely canceled out.

${x_2}^2(r+y_1-c_1)e^{2d} = (r-y_1+c_1)(r+x_1-c_1)^2$

$y_1[{x_2}^2e^{2d}+(r+x_1-c_1)^2] = (r+c_1)(r+x_1-c_1)^2-{x_2}^2(r-c_1)e^{2d}$

$y_1 = \frac{(r+c_1)(r+x_1-c_1)^2-{x_2}^2(r-c_1)e^{2d}}{{x_2}^2e^{2d}+(r+x_1-c_1)^2}$

Now that we know $y_1$, we can easily find $y_2$ with $y_2 = \sqrt{r^2-({y_1}-c_1)^2}$.

We will also need to find the change in orientation.

Let $\theta_0$ be the initial angle and $\theta_1$ be the final angle.

$\sin \theta_0 = \frac{x_2-c_2}{r}$

$= \frac{x_2}{r}$

$\cos \theta_0 = \frac{x_1-c_1}{r}$

$= -\frac{c_1}{r}$

Similarly, $\sin \theta_1 = \frac{y_2}{r}, \cos \theta_1 = \frac{y_1-c_1}{r}$.

We can simply take $\theta_1-\theta_0$ as the angle, but this will require trigonometry to get the angle, and then trigonometry later to do the rotation with it. It may be more optimal to just calculate $\sin \Delta\theta$ and $\cos \Delta\theta$ using angle sums.

$\sin(\theta_1-\theta_0) = \sin\theta_1 \cos\theta_0 - \cos\theta_1 \sin\theta_1$

$\cos(\theta_1-\theta_0) = \cos\theta_0 \cos\theta_1 - \sin\theta_0 \sin\theta_1$

If $z_0 = 0$ so the vector is pointing straight up or down, we have:

$z_2 = \ln\frac{y_2}{x_2}$

$e^{z_2} = \frac{y_2}{x_2}$

$y_2 = x_2e^{z_2}$

Clearly, $y_1$ remains constant and there is no rotation.

\end{document}
