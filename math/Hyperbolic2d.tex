\subsection{2d Hyperbolic Geometry}

\subsubsection{Finding the direction and distance from one point to another}

We will be using the half plane model: $\textbf{H}^2 \mapsto \{(x_1,x_2):x_2 > 0\}$.

Given points $\textbf{x} = (x_1,x_2)$ and $\textbf{y} = (y_1,y_2)$, we need to find the vector $\textbf{v}$ at $\textbf{x}$ that is tangent to the geodesic from $\textbf{x}$ to $\textbf{y}$, such that $\|\textbf{v}\|$ is the length of the geodesic arc between those points.

Assuming $x_1 \neq y_1$, the geodesic arc between $\textbf{x}$ and $\textbf{y}$ is mapped to an arc of a circle on the half-plane with its center on the $x$-axis. Consider this circle.

Let $\textbf{c} = (c_1,c_2)$ be the center of the circle, and $r$ be the radius. We know $c_2 = 0, \|\textbf{x}-\textbf{c}\| = \|\textbf{y}-\textbf{c}\| = r$.

First, let us solve for $c_1$.

We have $$(x_1-c_1)^2+{x_2}^2 = (y_1-c_1)^2+{y_2}^2 = r^2.$$

\ignore{
So, $${x_1}^2-2x_1c_1+c_1^2+{x_2}^2 = {y_1}^2-2y_1c_1+c_1^2+{y_2}^2$$

and $${x_1}^2-2x_1c_1+{x_2}^2 = {y_1}^2-2y_1c_1+{y_2}^2$$

therefore $$2(x_1-y_1)c_1 = {x_1}^2+{x'_2}^2-{y_1}^2-{y_2}^2$$
}

Solving for 

$$c_1 = \frac{({x_1}^2+{x_2}^2)-({y_1}^2+{y_2}^2)}{2(x_1-y_1)}$$

$$= \frac{\|\textbf{x}\|^2-\|\textbf{y}\|^2}{2(x_1-y_1)}$$

We can use this value of $c_1$ to compute $r$: $r = \sqrt{(x_1 - c_1)^2 + x_2^2}$

The vector $\textbf{v}$ is tangent to the circle, which means that it is at a right angle to the direction to $\textbf{c}$, which works out to be $\mat{0}{-1}{1}{0}(\textbf{c}-\textbf{x}) = \vect{x_2}{c_1-x_1}$. We then normalize this to $\displaystyle{\frac{(x_2,c_1-x_1)}{\sqrt{x_2^2+(c_1-x_1)^2}}}$.

The distance along the geodesic can be calculated using the angle of the arc between $\textbf{x}$ and $\textbf{y}$ on the half plane. Assuming $\textbf{y}$ is counterclockwise of $\textbf{x}$, the length of the geodesic arc is:

$$\int_{\theta_1}^{\theta_2} \frac{\sqrt{(\frac{d}{d\theta}r\sin\theta)^2+(\frac{d}{d\theta}r\cos\theta)^2}}{r\sin\theta} d\theta$$

$$= \int_{\theta_1}^{\theta_2} \frac{\sqrt{(r\cos\theta)^2+(-r\sin\theta)^2}}{r\sin\theta} d\theta$$

$$= \int_{\theta_1}^{\theta_2} \frac{r}{r\sin\theta} d\theta$$

$$= \int_{\theta_1}^{\theta_2} \csc\theta d\theta$$

$$= -\ln\left|\frac{\csc\theta_2+\cot\theta_2}{\csc\theta_1+\cot\theta_1}\right|$$

$$= \ln\left|\frac{\csc\theta_1+\cot\theta_1}{\csc\theta_2+\cot\theta_2}\right|.$$

We can substitute $$\csc\theta_1 = \frac{r}{x_2}, \cot\theta_1 = \frac{x_1-c_1}{x_2}, \csc\theta_2 = \frac{r}{y_2}, \cot\theta_2 = \frac{y_1-c_1}{y_2}.$$

Hence, the distance is $$\ln\left|\frac{\frac{r}{x_2}+\frac{x_1-c_1}{x_2}}{\frac{r}{y_2}+\frac{y_1-c_1}{y_2}}\right| = \ln\left|\frac{\frac{r+x_1-c_1}{x_2}}{\frac{r+y_1-c_1}{y_2}}\right| = \ln\left|\frac{y_2(r+x_1-c_1)}{x_2(r+y_1-c_1)}\right|$$

We already know $x_2$ and $y_2$ are greater than $0$. Since $\|\textbf{c}-\textbf{x}\| = \|\textbf{c}-\textbf{y}\| = r$, clearly $|c_1-x_1| < r$ and $|c_1-y_1| < r$, which means $c_1-x_1 < r$ and $c_1-y_1 < r$, so $r+x_1-c_1 > 0$ and $r+y_1-c_1 > 0$. So $\displaystyle\frac{y_2(r+x_1-c_1)}{x_2(r+y_1-c_1)} \geq 0$. Thus, we may remove the absolute value, and the distance is equal to $\displaystyle\ln\frac{y_2(r+x_1-c_1)}{x_2(r+y_1-c_1)}$

If the path is clockwise instead of counterclockwise, the integral given is opposite the direction of motion, so the result is negative of the distance: $$-\ln\frac{y_2(r+x_1-c_1)}{x_2(r+y_1-c_1)}.$$ In general, the distance is $$\left|\ln\frac{y_2(r+x_1-c_1)}{x_2(r+y_1-c_1)}\right|.$$

If $x_1 = y_1$, the geodesic from $\textbf{x}$ to $\textbf{y}$ is a vertical line, so the problem is easier. The direction of $\textbf{v}$ is $(0,1)$, and the distance is $$\int_{x_2}^{y_2} \frac{1}{t} dt = \ln\frac{y_2}{x_2}.$$ This gives a vector of $$\left(0,\ln\frac{y_2}{x_2}\right).$$

\subsubsection{Finding the point a given distance in a given direction from another}

Given initial point $\textbf{x} = (x_1,x_2)$ and vector $\textbf{v} = (v_1,v_2)$, we must find the endpoint $\textbf{y}$ of the geodesic arc starting at $\textbf{x}$ of length $\|\textbf{v}\|$ in the direction of $\textbf{v}$.

Suppose $v_1 \neq 0$. This means that the geodesic is an arc of a circle, instead of a vertical line segment. The center of the circle is on a line perpendicular to the tangent vector.

The Euclidean line tangent to the circle is $\textbf{x}+t\textbf{v}$, so the center of the circle is on $$\textbf{x}+t\mat{0}{-1}{1}{0}\textbf{v} = (x_1-tv_2,x_2+tv_1).$$

This line intersects with the $x$ axis when $x_2+tv_1 = 0$.

Solving for $t$, $$t = -\frac{x_2}{v_1}.$$

This implies $$c_1 = x_1-tv_2 = x_1+\frac{x_2v_2}{v_1},$$ and as before, $c_2 = 0$.

Now we can find $r = \|\textbf{x}-\textbf{c}\| = \sqrt{(x_1 - c_1)^2 + x_2^2}$.

%If were letting $x_1 = 0$, that's just $r^2 = c_1^2+{x_2}^2$.

Now that we have the circle, it's just a matter of finding the point at the right distance.

From our calculation above, we know distance, $$d = \ln\frac{y_2(r+x_1-c_1)}{x_2(r+y_1-c_1)}.$$

$$e^d = \frac{y_2(r+x_1-c_1)}{x_2(r+y_1-c_1)}$$

$$x_2(r+y_1-c_1)e^d = y_2(r+x_1-c_1)$$

$${x_2}^2(r+y_1-c_1)^2e^{2d} = {y_2}^2(r+x_1-c_1)^2$$

From the equation of a circle, we know $$(y_1-c_1)^2 + {y_2}^2 = r^2,$$ so $${y_2}^2 = r^2 - (y_1-c_1)^2.$$

Substituting this in,

$${y_2}^2(r+x_1-c_1)^2 = [r^2-(y_1-c_1)^2](r+x_1-c_1)^2$$

$$= [r-(y_1-c_1)][r+(y_1-c_1)](r+x_1-c_1)^2$$

$$= (r-y_1+c_1)(r+y_1-c_1)(r+x_1-c_1)^2.$$

Since $y_0 > 0$ and $r = \|\textbf{y}-\textbf{c}\|$, it must be the case that $|y_1-c_1| < r$ so $r+y_1-c_1 \neq 0$ and can be safely cancelled out of the equation $${x_2}^2(r+y_1-c_1)^2e^{2d} = (r-y_1+c_1)(r+y_1-c_1)(r+x_1-c_1)^2.$$

We obtain $${x_2}^2(r+y_1-c_1)e^{2d} = (r-y_1+c_1)(r+x_1-c_1)^2.$$

Solving for $y_1$, we get $$y_1[{x_2}^2e^{2d}+(r+x_1-c_1)^2] = (r+c_1)(r+x_1-c_1)^2-{x_2}^2(r-c_1)e^{2d},$$

so $$y_1 = \frac{(r+c_1)(r+x_1-c_1)^2-{x_2}^2(r-c_1)e^{2d}}{{x_2}^2e^{2d}+(r+x_1-c_1)^2}.$$

Now that we know $y_1$, we can easily find $y_2$ with $y_2 = \sqrt{r^2-({y_1}-c_1)^2}$.

We will also need to find the change in orientation when moving along this geodesic as a parallel transport.

%First, let us cover the case where $v_1 \neq 0$ and the geodesic is a circle.

Let $\theta_0$ be the angle of the normal to the geodesic at $\textbf{x}$ with respect to half plane coordinates.

Now we need to parallel transport this normal along the geodesic to $\textbf{y}$, and let $\theta_1$ be the angle of the result with respect to half plane coordinates.

Since we are performing a parallel transport across a geodesic, the parallel transport of a normal is also normal to the geodesic, but at $\textbf{y}$ instead of $\textbf{x}$.

Note that since the geodesic is a circle, the coordinates for the normal vector are the same as the position of $\textbf{x}$ relative to the center of the circle. This makes it easier to compute the angles.

First, let us compute $\sin\theta_0$ and $\cos\theta_0$ in terms of $\textbf{x}, \textbf{c},$ and $r$.

$$\sin \theta_0 = \frac{x_2-c_2}{r} = \frac{x_2}{r}$$

$$\cos \theta_0 = \frac{x_1-c_1}{r}$$

Similarly, $$\sin \theta_1 = \frac{y_2}{r}, \cos \theta_1 = \frac{y_1-c_1}{r}.$$

In order to computer a rotation, we need a rotation matrix which is computed with $\sin \Delta\theta$ and $\cos \Delta\theta$. Rather than solving for $\theta_0$ and $\theta_1$ with trigonometry, and finding the sin and the cosine of the difference, we can decrease the necessary computations by finding $\sin \Delta\theta$ and $\cos \Delta\theta$ with angle sums.

$$\sin(\theta_1-\theta_0) = \sin\theta_1 \cos\theta_0 - \cos\theta_1 \sin\theta_1$$

$$\cos(\theta_1-\theta_0) = \cos\theta_0 \cos\theta_1 - \sin\theta_0 \sin\theta_1$$

We use this to build the rotation matrix $$\mat{\cos \Delta\theta}{-\sin \Delta\theta}{\sin \Delta\theta}{\cos \Delta\theta},$$ which was what we wanted.

If $v_1 = 0$ and the geodesic is a vertical line, so the vector $\textbf{v}$ is pointing straight up or down, we have:

$$v_2 = \ln\frac{y_2}{x_2}$$

$$e^{v_2} = \frac{y_2}{x_2}$$

$$y_2 = x_2e^{v_2}$$

Clearly, $y_1 = x_1$ and there is no rotation.

%Portals in $3$-wormholes will be represented by $S^2$ slices of $S^2 \times \mathbb{R}$. The intersection with the $S^1 \times \mathbb{R}$ slice, when mapped to the upper half-plane, becomes a radial line. As such, it is necessary to find where geodesics in $\mathbb{H}^2$ intersect with these radial lines.

