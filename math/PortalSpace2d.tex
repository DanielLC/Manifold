\documentclass[12pt]{amsart}
\usepackage{a4}
\usepackage{amsmath,amssymb,amsthm}
\usepackage{multicol}
\usepackage{verbatim}
\newcommand{\sgn}{\mathop{\mathrm{sgn}}}
\newcommand{\ignore}[1]{}
\newcommand{\mat}[4]{\left(\begin{array}{ccc} #1 & #2 \\#3 & #4 \end{array} \right)}
\newcommand{\vect}[2]{\left(\begin{array}{ccc} #1 \\#2 \end{array} \right)}
\newcommand{\matc}[9]{\left(\begin{array}{ccc} #1 & #2 & #3 \\#4 & #5 & #6 \\#7 & #8 & #9 \end{array} \right)}
%\newcommand{\matd}[16]{\left(\begin{array}{ccc} #1 & #2 & #3 & #4 \\#5 & #6 & #7 & #8 \\#9 & #10 & #11 & #12 \\#13 & #14 & #15 & #16 \end{array} \right)}
\begin{document}

Consider the quotient space in $\mathbb{H}^2$ made by identifying two ultraparallel lines so that the intersections with their mutual perpendiculars are identified, and they both go in the same direction.

Suppose WLOG that the half-plane model is used, and the identified lines are concentric semi-circles centered on the origin.

Consider the automorphism subgroup generated by scaling the model linearly. This subgroup is homomorphic to $\mathbb{R}$. Taking the quotient by identifying the two lines results in a quotient of the automorphism subgroup homomorphic to $S^1$, showing that this results in a surface of revolution.

Note that wormholes are not all isomorphic. There is one variable that needs to be tracked: the distance between the identified lines. Let's call this distance $k$.

The surface can be mapped to $S^2 \times \mathbb{R}$ by mapping an angle of $\log\|\textbf{x}\|\frac{2\pi}{k}$ to $S^2$ where $\|x\|$ is taken under the Euclidean metric. This leaves $\frac{x}{\|x\|}$ to be mapped to $\mathbb{R}$. It doesn't matter much what is used. I am currently using $\arctan\frac{x_2}{x_1}$, though just using $\frac{x_2}{x_1}$ may be more efficient.

%Vectors must still be rotated.

\end{document}
