\documentclass[12pt]{amsart}
\usepackage{a4}
\usepackage{amsmath,amssymb,amsthm}
\usepackage{multicol}
\usepackage{verbatim}
\newcommand{\sgn}{\mathop{\mathrm{sgn}}}
\newcommand{\ignore}[1]{}
\newcommand{\mat}[4]{\left(\begin{array}{ccc} #1 & #2 \\#3 & #4 \end{array} \right)}
\newcommand{\vect}[2]{\left(\begin{array}{ccc} #1 \\#2 \end{array} \right)}
\newcommand{\matc}[9]{\left(\begin{array}{ccc} #1 & #2 & #3 \\#4 & #5 & #6 \\#7 & #8 & #9 \end{array} \right)}
%\newcommand{\matd}[16]{\left(\begin{array}{ccc} #1 & #2 & #3 & #4 \\#5 & #6 & #7 & #8 \\#9 & #10 & #11 & #12 \\#13 & #14 & #15 & #16 \end{array} \right)}
\begin{document}

\title{2-Wormhole}
\maketitle

Consider the quotient space of $\mathbb{H}^2$ generated by quotienting out by a loxodromic transformation. We can set the half-plane model $\mathbb{H}^2 = \{(x,y):y>0\}$ so that the axis that is translated along is the $x=0$ geodesic. The loxodromic transformation thus becomes $(x,y) \mapsto e^k(x,y)$ where $k$ is the hyperbolic distance translated.

The resulting space is a $(1,1)$-surface of revolution. In particular, the subgroup of the isomorphism group made up by the loxodromic transformations in that direction by arbitrary distances is $S^1$.

This quotient space can be mapped to $S^1 \times \mathbb{R}$ by setting the $S^1$ component to $\frac{2\pi}{k}\log\sqrt{x^2+y^2}$ and the $\mathbb{R}$ component to $\frac{y}{x}$. This is necessary to allow it to be generalized to an $(n,1)$-surface of revolution.

%Vectors must still be rotated.

In addition to mapping the points, we have to be able to convert vectors between the two. The vectors used in the half plane model are set so that $(0,1)$ points to the point at $(0,\infty)$ and $(1,0)$ is $90^\circ$ clockwise of this.

For the $2$-Wormhole, at $(x,y)$, $(0,1)$ should point to $(x,\infty)$ and $(1,0)$ should point to $(x+\epsilon,y)$.

Then a miracle happened.

Let $\theta = \arctan\frac{y}{x}$.

$M = M^{-1} = \mat{-\sin\theta}{\cos\theta}{\cos\theta}{\sin\theta}$

Multiplying by this matrix will convert vectors between the two coordinate systems.

\end{document}
