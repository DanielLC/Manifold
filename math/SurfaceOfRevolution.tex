\subsection{Surface Of Revolution}

In a $2$-dimensional surface of revolution, a curve is rotated around an axis and each point traces out a circle. This results in a surface that is preserved under rotation. That is to say, it has the symmetries of a circle. A surface of revolution can easily be generalized to higher dimensions by rotating a curve around an axis so each point traces an $n$-sphere in $\mathbb{R}^{n+2}$.

My program relies largely on intrinsic geometry, so it would be helpful to modify this definition to use the intrinsic geometry of a manifold. Rather than looking at how the manifold is formed, I will look at the symmetries it inherits.

Let $G \leq \Aut(S^n \times \mathbb{R})$ be the subgroup of automorphisms on $S^n \times \mathbb{R}$ that fix the $\mathbb{R}$ coordinate.

A possible generalization of a $n$-surface of revolution as a manifold $M$ such that there exists a homeomorphism $f:S^{n-1} \times \mathbb{R} \to M$ that preserves the automorphisms in $G$.

This definition introduces a few problems. For example, $\mathbb{S}^2$ would not be considered a surface of revolution, since there is no homeomorphism from $\mathbb{S}^1 \times \mathbb{R}$ to $S^2$. We can fix this by not requiring $f$ to be injective.

Instead, I define an $n$-surface of revolution to be a manifold $M$ such that there exists a continuous function $f:S^{n-1} \times \mathbb{R} \to M$ where given any automorphism $g$ in $G$, $f(x) \mapsto f \circ g(x)$ is well-defined and is an automorphism.

In Section \ref{Wormhole}, we discussed working with Wormhole. This was a special case of a method for working with $3$-surfaces of revolution.

We can reduce the problem of working on a $3$-surface of revolution (or even an $n$-surface of revolution) to working on a corresponding $2$-surface of revolution.

Consider a point $x$ on a $3$-manifold $U$ that is a surface of revolution and a vector $v$ in the tangent space $U_x$.

Consider the component $v_0$ of $v$ along $S^2$.

Consider the great circle made by extending $x$ to a geodesic in the direction of $v_0$.

We can reflect the spherical coordinate across this great circle. This is an automorphism on $S^2 \times \mathbb{R}$, and is therefore an automorphism on $U$.

By symmetry, the geodesic made by extending $x$ in the $v$ direction must stay on this slice of $U$.

This reduces finding the geodesic on a $3$-surface of revolution to finding one on a $2$-surface of revolution.

%In order to simplify working with $3$-surfaces of revolution, we can simplify the problem to working with a $2$-dimensional analogue.

%Given two points in the $3$-surface of revolution, we can take a $2$-dimensional slice which is the $2$-dimensional analogue. Once we find the vector corresponding to the distance and direction from one point to another in the $2$-dimensional slice, we can map the vector back into the $3$-surface of revolution.

\bigskip

\subsubsection{Finding the vectors from a point:}

\bigskip

Rather than mapping a sphere to $\mathbb{R}^2$, we can embed it in $\mathbb{R}^3$. Given points $\textbf{x} = (\textbf{x}_1, x_2), \textbf{y} = (\textbf{y}_1, y_2) \in S^2 \times \mathbb{R}$, let $\textbf{v}$ be the unit vector in the direction of $\textbf{y}_1$ from $\textbf{x}_1$. You can find this by looking at $\textbf{x}_1$ and $\textbf{y}_1$ as vectors in $\mathbb{R}^3$, taking the projection of $\textbf{y}_1$ perpendicular to $\textbf{x}_1$, and normalizing.

Let $\theta$ be the angle between $\textbf{x}_1$ and $\textbf{y}_1$, so $\theta = \arccos\left<\textbf{x}_1,\textbf{y}_1\right>$.

Now we take the points $\textbf{x}' = (0,x_4)$ and $\textbf{y}' = (\theta, y_4)$ in $S^1 \times \mathbb{R}$.

Let $\textbf{z}'$ be a vector between them.

Let the $S^2$ component of $\textbf{z}$ be $z'_0\textbf{v}$ and the $\textbf{R}$ component be $z'_1$.

$\textbf{z}$ is a vector between the two points.

\bigskip

\subsubsection{Finding a point from a vector:}

\bigskip

Given point $\textbf{x}$ and vector $\textbf{z}$,

Let $\textbf{x}' = (0,x_2), \textbf{z}' = (\|\textbf{z}_1\|,z_2)$.

Let $\textbf{v} = \|\textbf{z}_1\|$.

Find $\textbf{y}'$ with the two-dimensional version.

$\textbf{y} = (\textbf{x}_1\cos y'_1+\textbf{v}\sin y'_1, y'_2)$.

In order to find the rotation, we must work in a more relevant basis. We can do this by commuting with the appropriate matrix.

$\textbf{e}_1 \mapsto \textbf{e}_1, \textbf{v} \mapsto \textbf{e}_2, \textbf{x}_1 \mapsto \textbf{e}_3, \textbf{x}_1 \times \textbf{v} \mapsto \textbf{e}_4$

$(\textbf{e}_1, \textbf{v},\textbf{x}_1,(\textbf{x}_1 \times \textbf{v}))^{-1}$

$= (\textbf{e}_1, \textbf{v},\textbf{x}_1,(\textbf{x}_1 \times \textbf{v}))^T$, since it's a rotation matrix.

First, we use the rotation of the two-dimensional version to find the rotation between $\textbf{e}_1$ and $\textbf{v}$.

%$\mat{\cos\theta}{-\sin\theta}{\sin\theta}{\cos\theta}$

$\left(\begin{array}{cccc} \cos\theta & -\sin\theta & 0 & 0 \\ \sin\theta & \cos\theta & 0 & 0 \\ 0 & 0 & 1 & 0 \\ 0 & 0 & 0 & 1 \end{array} \right)$

Next, we have to deal with the fact that the $S^2$ component rotates.

Let $\phi = y'_2-x'_2$.

$$\left(\begin{array}{cccc} 1 & 0 & 0 & 0 \\ 0 & \cos\phi & -\sin\phi & 0 \\ 0 & \sin\phi & \cos\phi & 0 \\ 0 & 0 & 0 & 1 \end{array} \right)$$

Putting this all together, we get:

$$(\textbf{e}_1,\textbf{z},\textbf{x},(\textbf{x} \times \textbf{z}))
\left(\begin{array}{cccc} 1 & 0 & 0 & 0 \\ 0 & \cos\phi & -\sin\phi & 0 \\ 0 & \sin\phi & \cos\phi & 0 \\ 0 & 0 & 0 & 1 \end{array} \right)
\left(\begin{array}{cccc} 1 & 0 & 0 & 0 \\ 0 & \cos\phi & -\sin\phi & 0 \\ 0 & \sin\phi & \cos\phi & 0 \\ 0 & 0 & 0 & 1 \end{array} \right)
(\textbf{e}_1,\textbf{z},\textbf{x},(\textbf{x} \times \textbf{z}))^T$$

\bigskip

\subsubsection{Portals:}

The intersection tells me the orientation as embedded in $\mathbb{R}^3$. I need to convert that to $\mathbb{R} \times S^2 \subseteq \mathbb{R}^4$.

First, I just expand the matrix $M$ to $\mat{M}{0}{0}{1}$ to put it in $\mathbb{R}^3 \subseteq \mathbb{R}^4$.

Now I need to reflect it twice.

First, I will reflect along the $t$-axis, to get $\mat{M}{0}{0}{-1}$. This doesn't change anything, since that vector was orthogonal to reality, but it does guarantee that the final orientation will have the same sign.

Now I need to reflect between $(0,0,0,1)$ and $(v,0)$ where $v \in S^2$ is the position of the vector.

Let $w$ be a vector I am moving with this, and $R$ be the reflection matrix.

$Rw = w - ((v,0)-(0,0,0,1))\left<(v,0)-(0,0,0,1),w\right>$

$= w - (v,-1)\left<(v,-1),w\right>$

$= Iw - (v,-1)(v,-1)^Tw$

$= (I - (v,-1)(v,-1)^T)w$

$R = I - (v,-1)(v,-1)^T$

This means that the final orientation is $R\mat{M}{0}{0}{-1}$.

Or rather, it would be, except that the real coordinate is the first coordinate, not the last. I just need to cycle it through with the matrix

$S = \left(\begin{array}{cccc}
		0 & 0 & 0 & 1 \\
		1 & 0 & 0 & 0 \\
		0 & 1 & 0 & 0 \\
		0 & 0 & 1 & 0
	\end{array} \right)$

Giving me

$SR\mat{M}{0}{0}{1}$

\ignore{
Let $\textbf{w}$ be the direction you're moving from your own point of reference.

$M\textbf{w}$ is the direction you'd end up going in Euclidean geometry.

In the surface of revolution, the program pads a $0$ at the end, signifying that you are not moving at all perpendicular to reality. This means the direction from your point of reference is $(\textbf{w},0)$.

If you're lucky enough that $M\textbf{w}$ is tangent to the portal, then you only have to deal with changing coordinates, and the direction you'll end up going is $(0,M\textbf{w})$.

Thus, you can think of it as $(\textbf{w},0) \mapsto (0,M\textbf{w}), (\textbf{0},1) \mapsto (1,\textbf{0})$, which is represented by the linear transformation $\mat{\textbf{0}^T}{1}{M}{\textbf{0}}$.

If $M\textbf{w}$ is perpendicular to the portal, $(w,0)$ should move to $(\|w\|,\textbf{0})$. The previous transformation sends it to $(0,M\textbf{w})$ instead. We need to reflect between these two, while preserving anything perpendicular to them.

Assuming for simplicity that $M\textbf{w}$ is unit. Since we're looking at a unit vector perpendicular to the portal, we've already narrowed it down to a specific vector, so we can just call it $\textbf{v}$. We need to reflect between $(1,\textbf{0})$ and $(0,\textbf{v})$.

This gives $R = I - (-1,\textbf{v})(-1,\textbf{v})^T$.

Thus, the final matrix is $R\mat{\textbf{0}^T}{1}{M}{\textbf{0}}$.
}