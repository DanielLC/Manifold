\section{Surface Of Revolution}

In a $2$-dimensional surface of revolution, a curve is rotated around an axis and each point traces out a circle. This results in a surface that is preserved under rotation. That is to say, it has the symmetries of a circle. A surface of revolution can easily be generalized to higher dimensions by rotating a curve around an axis so each point traces an $n$-sphere in $\mathbb{R}^{n+2}$.

My program relies largely on intrinsic geometry, so it would be helpful to modify this definition to use the intrinsic geometry of a manifold. Rather than looking at how the manifold is formed, I will look at the symmetries it inherits.

Let $G \leq \Aut(\mathbb{R} \times S^n)$ be the subgroup of automorphisms on $S^n \times \mathbb{R}$ that fix the $\mathbb{R}$ coordinate.

A possible generalization of a surface of revolution is a manifold $M$ such that there exists a homeomorphism $f:\mathbb{R} \times S^{n-1} \to M$ that preserves the automorphisms in $G$.

This definition introduces a few problems. For example, $\mathbb{S}^2$ would not be considered a surface of revolution, since there is no homeomorphism from $\mathbb{R} \times \mathbb{S}^1$ to $S^2$. We can fix this by not requiring $f$ to be injective.

Instead, I define an $n$-surface of revolution to be a manifold $M$ such that there exists a continuous function $f:\mathbb{R} \times S^{n-1} \to M$ where given any automorphism $g$ in $G$, $f(x) \mapsto f \circ g(x)$ is well-defined and is an automorphism.

In Section \ref{Wormhole}, we discussed working with Wormhole. This was a special case of a method for working with $3$-surfaces of revolution.

We can reduce the problem of working on a $3$-surface of revolution (or even an $n$-surface of revolution) to working on a corresponding $2$-surface of revolution.

Consider a point $x$ on a $3$-manifold $M$ that is a surface of revolution and a vector $\textbf{v}$ in the tangent space $M_x$. Consider the component $v_1$ of $\textbf{v}$ along the $S^2$ cross-section of $M$. Consider the great circle, $C$, made by extending a geodesic from $x$ in the direction of $v_1$.

We can reflect the spherical coordinate across $C$. This is an automorphism on $\mathbb{R} \times S^2$ that fixes the $\mathbb{R}$ coordinate, and therefore corresponds to an automorphism on $M$. By symmetry, the geodesic extended from $x$ in the $\textbf{v}$ direction must stay on the slice of $M$ fixed by this automorphism. Note that this slice is the image of the coordinates $C \times \mathbb{R}$. Since $C$ is isometric to $S^1$ it inherits the automorphisms from $G$, so this slice is a $2$-surface of revolution. This reduces finding the geodesic on a $3$-surface of revolution to finding one on a $2$-surface of revolution.

This shows that any geodesic must stay on such a slice of $M$. Given two points on $M$, if we wish to find a geodesic between them, it must fall on a slice that connects them. As long as the spherical coordinates of the points are not the same or antipodal, there is exactly one such slice, so to find the geodesic on $M$ that connects them, we only need to find and lift the corresponding geodesic on this slice.

\begin{remark}
When dealing with a vector perpendicular to the $S^2$ slice, or two points who share an $S^2$ component, or two points whose $S^2$ components are antipodal, I pick one of these geodesic slices arbitrarily. Everything regarding existence of a geodesic still applies as above. When finding a geodesic connecting two points, uniqueness does not hold. Since this almost never happens, and there's no good way to deal with uncountably many geodesics in my application, I consider it sufficient for this case to not crash the program.
\end{remark}

%In order to simplify working with $3$-surfaces of revolution, we can simplify the problem to working with a $2$-dimensional analogue.

%Given two points in the $3$-surface of revolution, we can take a $2$-dimensional slice which is the $2$-dimensional analogue. Once we find the vector corresponding to the distance and direction from one point to another in the $2$-dimensional slice, we can map the vector back into the $3$-surface of revolution.

\bigskip

\subsection{Finding the vectors from a point:}

\bigskip

We will be using $f:\mathbb{R} \times S^2 \to M$ above as a coordinate system. Since the computer only deals with reals, we will embed $S^2$ in $\mathbb{R}^3$. Given points $\textbf{x} = (x_1, \textbf{x}_2), \textbf{y} = (y_1, \textbf{y}_2) \in \mathbb{R} \times S^2$, let $\textbf{w}$ be the unit vector in the direction of $\textbf{y}_2$ from $\textbf{x}_2$. You can find this by looking at $\textbf{x}_2$ and $\textbf{y}_2$ as vectors in $\mathbb{R}^3$, taking the projection of $\textbf{y}_2$ perpendicular to $\textbf{x}_2$, and normalizing.

Let $\theta$ be the angle between $\textbf{x}_2$ and $\textbf{y}_2$, so $\theta = \arccos\left<\textbf{x}_2,\textbf{y}_2\right>$. Now we take the points $\textbf{x}' = (x_1, 0)$ and $\textbf{y}' = (y_1, \theta)$ in $\mathbb{R} \times S^1$. Let $\textbf{v}' \in (\mathbb{R} \times S^1)_\textbf{x}$ be a vector at $\textbf{x}'$ pointing to $\textbf{y}'$ of length equal to their distance; finding this is the problem that we are reducing to. Assume we have found $\textbf{v}'$, we can let $\textbf{v} \in (\mathbb{R} \times S^2)_\textbf{x}$ be $(v'_1, v'_2\textbf{w})$. Then $\textbf{v}$ is the vector that we wanted.


% the $S^2$ component of $\textbf{v}$ be $v'_1\textbf{w}$ and the $\textbf{R}$ component be $v'_2$. $\textbf{v}$ is a vector between the two points.

\bigskip

\subsection{Finding a point from a vector:}

\bigskip

Given point $\textbf{x} = (x_1, \textbf{x}_2) \in \mathbb{R} \times S^2$ and vector $\textbf{v} = (v_1, \textbf{v}_2) \in (\mathbb{R} \times S^2)_\textbf{x} = \mathbb{R}_{x_1} \times S^2_{\textbf{x}_2}$, let $\textbf{x}' = (x_1,0), \textbf{v}' = (v_1,\|\textbf{v}_2\|)$, and let $\textbf{w} = \frac{\textbf{v}_2}{\|\textbf{v}_2\|}$. Use the solution for the two-dimensional case to find $\textbf{y}' = \exp_{\textbf{x}'}(\textbf{v}')$.

%Find the point $\textbf{y}' \in S^1 \times \mathbb{R}$ with the two-dimensional version.

Once we have $\textbf{y}'$, let $\textbf{y} \in \mathbb{R} \times S^2$ be the point whose $S^2$ coordinate is at an angle $\textbf{y}'_2$ from $\textbf{x}_2$ in the direction of $\textbf{w}$, and whose $\mathbb{R}$ coordinate is the $\mathbb{R}$ coordinate of $\textbf{y}'$. This works out to $(y'_1, \textbf{x}_2\cos y'_2+\textbf{w}\sin y'_2)$.

In order to find the change in orientation induced by the parallel transport, we must switch to a basis that is easier to work with. We can do this by conjugating with the appropriate matrix.

We will use the basis vectors $(\textbf{e}_1, \textbf{w}, \textbf{x}_1, (\textbf{x}_1 \times \textbf{w}))$. That list of vectors, taken as a matrix $C$, would transform a vector from the basis we want to the basis $(\textbf{e}_1,\textbf{e}_2,\textbf{e}_3,\textbf{e}_4)$. In order to transform to the basis we want, we take its inverse. Since it's an orthogonal basis, the inverse is the transpose, giving me $(\textbf{e}_1, \textbf{w},\textbf{x}_1,(\textbf{x}_1 \times \textbf{w}))^T = C^T$. In order to work with a vector in this basis, we need to transform it into this basis, apply whatever linear transformations we need, and then transform it back into the original basis. Thus, to transform by matrix $M$, we will compute $CMC^T$.

When we reduce to the two-dimensional case, the change in orientation is given by a rotation $\theta$. This is rotating between the real coordinate and a coordinate which is in the direction $\textbf{w}$. Since these are the first two coordinates of our new basis, we use $\theta$ to find the rotation matrix between $\textbf{e}_1$ and $\textbf{w}$, as follows:

%$\mat{\cos\theta}{-\sin\theta}{\sin\theta}{\cos\theta}$

$$\left(\begin{array}{cccc} \cos\theta & -\sin\theta & 0 & 0 \\ \sin\theta & \cos\theta & 0 & 0 \\ 0 & 0 & 1 & 0 \\ 0 & 0 & 0 & 1 \end{array} \right)$$

The $S^2$ component of a tangent vector rotates when moving along a great circle. Let $\phi = y'_2-x'_2$. The following matrix gives this rotation:

$$\left(\begin{array}{cccc} 1 & 0 & 0 & 0 \\ 0 & \cos\phi & -\sin\phi & 0 \\ 0 & \sin\phi & \cos\phi & 0 \\ 0 & 0 & 0 & 1 \end{array} \right)$$

Putting this all together, we get:

$$C
\left(\begin{array}{cccc} 1 & 0 & 0 & 0 \\ 0 & \cos\phi & -\sin\phi & 0 \\ 0 & \sin\phi & \cos\phi & 0 \\ 0 & 0 & 0 & 1 \end{array} \right)
\left(\begin{array}{cccc} \cos\theta & -\sin\theta & 0 & 0 \\ \sin\theta & \cos\theta & 0 & 0 \\ 0 & 0 & 1 & 0 \\ 0 & 0 & 0 & 1 \end{array} \right)
C^T$$

\bigskip

\subsection{Portals:}

In order to find the path of a geodesic that passes through a portal, I must first find where it intersects the portal, the orientation at this point, the direction the geodesic is moving in, and how much further it has to go. I put all of this information into an Intersection object, then I pass it to the space on the other side of the portal, and use it to create a geodesic with the corresponding position, orientation, direction, and length.

%TODO that should probably get its own section

The coordinates of the Intersection are treated as if the portal is embedded in Euclidean geometry. This makes using it trivial for Euclidean geometry. The coordinates of the point of intersection for Surface of Revolution are still trivial, since it's just the $\mathbb{S}^3$ coordinate. The orientation, however, is not. In this section we will discuss how to convert the orientation between these two coordinate systems.

The Intersection tells me the orientation as embedded in $\mathbb{R}^3$. I need to convert that to $\mathbb{R} \times S^2 \subseteq \mathbb{R}^4$.

We will change the basis from $\mathbb{R} \times S^2$ to $S^2 \times \mathbb{R}$ using 

$$S = \left(\begin{array}{cccc}
		0 & 0 & 0 & 1 \\
		1 & 0 & 0 & 0 \\
		0 & 1 & 0 & 0 \\
		0 & 0 & 1 & 0
	\end{array} \right)$$

We expand the matrix $M$ to $\mat{M}{0}{0}{1}$ to put it in $\mathbb{R}^3 \subseteq \mathbb{R}^4$.

Now we need to reflect it twice.

First, we will reflect along the $t$-axis, to get $\mat{M}{0}{0}{-1}$. This doesn't change anything, since that vector was orthogonal to reality, but it does guarantee that the final orientation will have the same sign as the original.

Now we need to reflect between $(0,0,0,1)$ and $(\textbf{v},0)$ where $\textbf{v} \in S^2$ is the position of the intersection. Let $\textbf{w}$ be a vector to which we are applying the orientation, and $R$ be the reflection matrix that reflects between these two.

\begin{align*}
R\textbf{w} & = \textbf{w} - ((\textbf{v},0)-(0,0,0,1))\left<(\textbf{v},0)-(0,0,0,1),w\right>\\
& = \textbf{w} - (\textbf{v},-1)\left<(\textbf{v},-1),\textbf{w}\right>\\
& = I\textbf{w} - (\textbf{v},-1)(\textbf{v},-1)^T\textbf{w}\\
& = (I - (\textbf{v},-1)(\textbf{v},-1)^T)\textbf{w}\\
R & = I - (\textbf{v},-1)(\textbf{v},-1)^T
\end{align*}

This means that the final orientation is $SR\mat{M}{0}{0}{-1}$.

\ignore{
Or rather, it would be, except that the real coordinate is the first coordinate, not the last. I just need to cycle it through with the matrix

$S = \left(\begin{array}{cccc}
		0 & 0 & 0 & 1 \\
		1 & 0 & 0 & 0 \\
		0 & 1 & 0 & 0 \\
		0 & 0 & 1 & 0
	\end{array} \right)$

Giving me

$SR\mat{M}{0}{0}{1}$
}

\ignore{
Let $\textbf{w}$ be the direction you're moving from your own point of reference.

$M\textbf{w}$ is the direction you'd end up going in Euclidean geometry.

In the surface of revolution, the program pads a $0$ at the end, signifying that you are not moving at all perpendicular to reality. This means the direction from your point of reference is $(\textbf{w},0)$.

If you're lucky enough that $M\textbf{w}$ is tangent to the portal, then you only have to deal with changing coordinates, and the direction you'll end up going is $(0,M\textbf{w})$.

Thus, you can think of it as $(\textbf{w},0) \mapsto (0,M\textbf{w}), (\textbf{0},1) \mapsto (1,\textbf{0})$, which is represented by the linear transformation $\mat{\textbf{0}^T}{1}{M}{\textbf{0}}$.

If $M\textbf{w}$ is perpendicular to the portal, $(w,0)$ should move to $(\|w\|,\textbf{0})$. The previous transformation sends it to $(0,M\textbf{w})$ instead. We need to reflect between these two, while preserving anything perpendicular to them.

Assuming for simplicity that $M\textbf{w}$ is unit. Since we're looking at a unit vector perpendicular to the portal, we've already narrowed it down to a specific vector, so we can just call it $\textbf{w}$. We need to reflect between $(1,\textbf{0})$ and $(0,\textbf{w})$.

This gives $R = I - (-1,\textbf{w})(-1,\textbf{w})^T$.

Thus, the final matrix is $R\mat{\textbf{0}^T}{1}{M}{\textbf{0}}$.
}