\documentclass[12pt]{amsart}
\usepackage{a4}
\usepackage{amsmath,amssymb,amsthm}
\usepackage{multicol}
\usepackage{verbatim}
\DeclareMathOperator*{\arctanh}{arctanh}
\DeclareMathOperator*{\sech}{sech}
\newcommand{\ignore}[1]{}
\newcommand{\mat}[4]{\left(\begin{array}{ccc} #1 & #2 \\#3 & #4 \end{array} \right)}
\newcommand{\vect}[2]{\left(\begin{array}{ccc} #1 \\#2 \end{array} \right)}
\newcommand{\matc}[9]{\left(\begin{array}{ccc} #1 & #2 & #3 \\#4 & #5 & #6 \\#7 & #8 & #9 \end{array} \right)}
\begin{document}

$E = G = \frac{1}{y^2}, F = 0$

$E_u = G_u = 0$

$E_v = G_v = -\frac{2}{y^3}$

$\Gamma^1_{11} = \frac{E_u}{2E}$

$= 0$

$\Gamma^1_{12} = \frac{E_v}{2E}$

$= -\frac{1}{y}$

$\Gamma^1_{22} = -\frac{G_u}{2E}$

$= 0$

$\Gamma^2_{11} = -\frac{E_v}{2G}$

$= \frac{1}{y}$

$\Gamma^2_{12} = \frac{G_u}{2G}$

$= 0$

$\Gamma^2_{22} = \frac{G_v}{2G}$

$= -\frac{1}{y}$

$\kappa_g = \sqrt{EG-F^2}[-\Gamma^2_{11}u'^3 + \Gamma^1_{22}v'^3 - (2\Gamma^2_{12} - \Gamma^1_{11})u'^2 v' + (2\Gamma^1_{12} - \Gamma^2_{22})u'v'^2 + u''v' - v''u'] \times (Eu'^2 + 2Fu'v' + Gv'^2)^{-3/2}$

$= \frac{1}{y^2}[-\frac{1}{y}u'^3 - \frac{1}{y}u'v'^2 + u''v' - v''u'] \times (\frac{1}{y^2}u'^2 + \frac{1}{y^2}v'^2)^{-3/2}$

The hypercycle can be parametrized as $\alpha(t) = (t \cos \theta,t \sin \theta)$ for some $\theta$.

$u' = \cos \theta, v' = \sin \theta, u'' = v'' = 0$

$\kappa_g = \frac{1}{y^2}(-\frac{1}{y}\cos^3 \theta - \frac{1}{y}\cos \theta \sin^2 \theta)(\frac{1}{y^2}\cos^2 \theta + \frac{1}{y^2}\sin^2 \theta)^{-3/2}$

$= \frac{1}{y^3}(-\cos^3 \theta - \cos \theta \sin^2 \theta)y^3$

$= -(\cos^3 \theta + \sin^2 \theta \cos \theta)$

$= -\cos\theta(\sin^2\theta + \cos^2\theta)$

$= -\cos \theta$

???

Circumference:

Let $k$ be the circumference of the bottleneck.

$C = \int_1^{e^k} \frac{1}{s\sin\theta} ds$

$= k\csc\theta$

???

Euclidean:

$C = \frac{2 \pi}{|\kappa_g|}$

$k\csc\theta = 2\pi\sec\theta$

$\tan\theta = \frac{2\pi}{k}$

???

Hyperbolic:

TODO: This needs to be scaled. At the default sizes, the wormhole cannot be attached to hyperbolic geometry.

A circle can be parameterized as $\alpha(\theta) = (\cos \theta, k + \sin \theta)$ for some $k$.

$u' = -\sin\theta, v' = \cos\theta$

$u'' = -\cos\theta, v'' = -\sin\theta$

$\kappa_g = \frac{1}{y^2}[-\frac{1}{y}u'^3 - \frac{1}{y}u'v'^2 + u''v' - v''u'] \times (\frac{1}{y^2}u'^2 + \frac{1}{y^2}v'^2)^{-3/2}$

$= \frac{1}{y^2}\left(\frac{1}{y}\sin^3\theta + \frac{1}{y}\sin\theta\cos^2\theta - \cos^2\theta - \sin^2\theta\right)\left(\frac{1}{y^2}\sin^2\theta + \frac{1}{y^2}\cos^2\theta\right)^{-3/2}$

$= \frac{1}{y^2}\left(\frac{1}{y}\sin\theta - 1\right)\left(\frac{1}{y^2}\right)^{-3/2}$

$= y\left(\frac{1}{y}\sin\theta - 1\right)$

$= \sin\theta - y$

$= k$

Suppose a circle has radius $r$:

$\frac{k+1}{k-1} = e^{2r}$

$k+1 = e^{2r}k-e^{2r}$

$(e^{2r}-1)k = e^{2r}+1$

$k = \frac{e^{2r}+1}{e^{2r}-1}$

$= \frac{1}{\tanh r}$

Now to find the circumference:

$C = \int_0^{2\pi} \frac{1}{k+\sin\theta} d\theta$

%(2*ArcTan[(1 + k*Tan[x/2])/Sqrt[-1 + k^2]])/ Sqrt[-1 + k^2]

$= \left.\frac{2\arctan \frac{1 + \tan\frac{x}{2}}{\sqrt{k^2-1}}}{\sqrt{k^2-1}}\right|_0^{2\pi}$, from Wolfram Integrator.

This function has period $2\pi$, but it's not continuous.

Every $2\pi$, $\tan\frac{x}{2}$ jumps from $\infty$ to $-\infty$.

$\frac{1 + \tan\frac{x}{2}}{\sqrt{k^2-1}}$ jumps from $\infty$ to $-\infty$.

$\arctan \frac{1 + \tan\frac{x}{2}}{\sqrt{k^2-1}}$ jumps from $\frac{\pi}{2}$ to $-\frac{\pi}{2}$.

$\frac{2\arctan \frac{1 + \tan\frac{x}{2}}{\sqrt{k^2-1}}}{\sqrt{k^2-1}}$ jumps from $\frac{\pi}{\sqrt{k^2-1}}$ to $-\frac{\pi}{\sqrt{k^2-1}}$.

$C = \frac{2\pi}{\sqrt{k^2-1}}$

$= \frac{2\pi}{\sqrt{\frac{1}{\tanh^2 r}-1}}$

$= \frac{2\pi\tanh r}{\sqrt{1-\tanh^2 r}}$

$= \frac{2\pi\tanh r}{\sech r}$

$= 2\pi\sinh r$

???

According to http://www.maths.gla.ac.uk/wws/cabripages/hyperbolic/circleformulae.html, $C = 2\pi\sinh r$

$r = \arctanh\frac{1}{\kappa_g}$

$C = 2\pi\sinh\arctanh\frac{1}{\kappa_g}$

$= 2\pi\frac{\frac{1}{\kappa_g}}{\sqrt{1-\frac{1}{\kappa_g^2}}}$

$= \frac{2\pi}{\sqrt{\kappa_g^2-1}}$

$= \frac{2\pi}{\sqrt{\cos^2\theta-1}}$

$= \frac{2\pi}{\sqrt{\cos^2\theta-1}}$


\end{document}
