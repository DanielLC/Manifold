\documentclass[12pt]{amsart}
\usepackage{a4}
\usepackage{amsmath,amssymb,amsthm}
\usepackage{multicol}
\usepackage{verbatim}
\usepackage{graphicx, subfigure}
\newcommand{\sgn}{\mathop{\mathrm{sgn}}}
\newcommand{\ignore}[1]{}
\newcommand{\mat}[4]{\left(\begin{array}{ccc} #1 & #2 \\#3 & #4 \end{array} \right)}
\newcommand{\vect}[2]{\left(\begin{array}{ccc} #1 \\#2 \end{array} \right)}
\newcommand{\matc}[9]{\left(\begin{array}{ccc} #1 & #2 & #3 \\#4 & #5 & #6 \\#7 & #8 & #9 \end{array} \right)}
\begin{document}

\title{Background}
\maketitle

\ignore{1.  Describe 3-manifold decompositions.
  A.  Prime decomposition theorem -- see Allen Hatcher's notes for statement and references.

%http://www.math.cornell.edu/~hatcher/3M/3M.pdf
}
\ignore{@Misc{•,
OPTkey = {•},
OPTauthor = {Allen Hatcher},
OPTtitle = {Notes on Basic 3-Manifold Topology},
OPThowpublished = {•},
OPTmonth = {•},
OPTyear = {•},
OPTnote = {•},
OPTannote = {•}
}}

The prime decomposition theorem states that a compact, connected, orientable 3-manifold can be decomposed into a connected sum of prime manifolds (manifolds that cannot be decomposed further, except by trivially removing a sphere) \cite{Kneser}, and that this is unique up to insertion or deletion of copies of $S^3$ \cite{Milnor}. The connected sum of two spaces is made by removing a ball from each and identifying their bounding spheres.
%What are the prime manifolds? What about manifolds that are not compact or are not orientable?
  
%  B.  Torus decomposition theorem -- see Allen Hatcher's notes for statement and references.

The torus decomposition theorem states that there is a minimal collection of disjointly embedded incompressible tori such that cutting along the edge of each yields components that are each either atoroidal or Seifert-fibered. It also states that this collection is unique up to isomorphism \cite{JSJ3} \cite{JSJ2} \cite{JSJ1} \cite{JSJ4}.

%Wikipedia has four references. Which do I use?

%  C.  Geometrization theorem.  (Wikipedia probably has statement, references?)

The geometrization theorem states that any 3-manifold can be decomposed canonically into submanifolds which each is a quotient space of one of the following eight geometries: $S^3, \mathbb{E}^3, \mathbb{H}^3, S^2 \times \mathbb{R}, \mathbb{H}^2 \times \mathbb{R}, \tilde{SL}(2,\mathbb{R}),$ Nil geometry, and Sol geometry. This is done by decomposing along the spheres given by the prime decomposition theorem and tori given by the torus decomposition theorem. The initial work was done by Grisha Perelman \cite{Perelman1} \cite{Perelman3} \cite{Perelman2}, and was later completed by other mathematicians \cite{Geometrization1} \cite{Geometrization2} \cite{Geometrization3}.

%  [D.  Your work:  take some of 8 geometries, glue together via "smooth" prime decomposition.]

I am working on a program that will allow smooth creation of a connected sum of several geometries as in the geometrization theorem, and visualization of the result. ``Smoothness'' means that the boundaries that are identified have the same curvatures in each surface. If they are not smoothly identified, then they will appear to have different curvatures from each side. As a result, a geodesic that barely intersects the border can have a very different path from one that barely misses.

For example, if we glue the outside of a sphere in Euclidean geometry to the outside of another such sphere in another copy of Euclidean geometry, the result looks like the portal is a reflective sphere, but with the reflection from the other geometry. This has effects such as blocking anything behind the sphere. This is impossible in a true manifold. Any point is visible from any other. My program will avoid this by using an intermediate geometry of non-constant curvature which contains spheres that can be glued to spaces of any constant curvature.

My program likely will not support torus decomposition, and thus will not be able to show every manifold. However, it is still more general than anything that has been reported previously, as described next.

%Additionally, I hope to make my program run real-time. While this is common for programs simulating spherical, Euclidean, and hyperbolic geometry, I do not believe it has been done for more sophisticated geometries, as there is not enough time to find geodesics numerically.

%Cone points?

%I also hope to add cone points to my program.


%2.  Visualizing 3-manifolds.
%  A.  Jeff Weeks website -- compact manifolds of constant curvature

Previous work has been done to visualize $S^3$ and $\mathbb{H}^3$ and quotient spaces thereof. In particular, Weeks has written a program that can view compact quotient spaces of $S^3, \mathbb{E}^3,$ and $\mathbb{H}^3$ \cite{CurvedSpaces}. He also wrote a paper that describes the process in depth \cite{CurvedSpacesPaper}.

Gunn and Maxwell made a video showing the complementary spaces of knots, most of which are quotient spaces of $\mathbb{H}^3$ \cite{NotKnot}. Gunn explains the techniques he uses in \cite{CharlieGunn} \cite{CharlieGunn2}.

Levy, Munzner and Mark Phillips wrote a geometry viewer called Geomview. Among its features is the ability to render non-Euclidean geometry \cite{Geomview}.

The main difference that will distinguish my work is the ability to create connected sums between spaces.

%  B.  Youtube movie

%http://www.spacetimetravel.org/wurmlochflug/wurmlochflug.html

I am unaware of any previous attempts to visualize general connected sums of geometries. However, attempts have been made to visualize a connected sum of two $\mathbb{E}^3$ spaces, with curvature near zero outside of a small wormhole.

Rune Johansen made a video that was designed to convey the idea of a wormhole \cite{runevision}. This video is primarily artistic in nature, and usees various tricks to make space look curved. There is no geometry that looks precisely like the video.

Corvin Zahn made a computer generated video precisely illustrating a wormhole \cite{spacetimetravel}. He uses a solution to Einstein's field equations that has been found previously \cite{WormholeSolution}, and simulated the camera moving through this wormhole. He did not provide specifics about the program used. He detailed what kind of manifold was used, but not how he used it. I have emailed him with questions regarding his implementation, but he has yet to respond.

Zahn's wormhole was made as one continuous geometry with everywhere negative curvature. While this is a much more interesting space than what my program will make, it's much harder to generalize. Adding a second wormhole between the spaces or a second wormhole leading to a third space would require redesigning the entire manifold. In contrast, my program will provide a compact manifold with boundary to connect spaces, so as long as the spheres they replace in Euclidean geometry do not touch, any number can be added easily to the same space.

Apparently, he used a ray tracing method and found the geodesics numerically. I intend to use rasterization and minimal amounts of numerical calculations in order to run the program in real time.

%  C.  Other?  E.g. old references on visualizing hyperbolic geometry?

%3.  Cone manifolds
%  A.  Define.  For example see definition in book "Three-dimensional orbifolds and cone manifolds" by Cooper, Hodgson, Kerckhoff.  Find other references there.
%  [B.  Note they are typically studied for manifolds of constant sectional curvature.  You will look at nonconstant.]

%Is this the same as the Sasakian manifold?

%https://en.wikipedia.org/wiki/Clairaut's_relation Good for solving surfaces of revolution numerically.

\bibliographystyle{amsplain}
\bibliography{Bibliography}

\end{document}
