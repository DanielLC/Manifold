\documentclass[12pt]{amsart}
\usepackage{a4}
\usepackage{amsmath,amssymb,amsthm}
\usepackage{multicol}
\usepackage{verbatim}
\usepackage{graphicx, subfigure}
\newcommand{\sgn}{\mathop{\mathrm{sgn}}}
\newcommand{\ignore}[1]{}
\newcommand{\mat}[4]{\left(\begin{array}{ccc} #1 & #2 \\#3 & #4 \end{array} \right)}
\newcommand{\vect}[2]{\left(\begin{array}{ccc} #1 \\#2 \end{array} \right)}
\newcommand{\matc}[9]{\left(\begin{array}{ccc} #1 & #2 & #3 \\#4 & #5 & #6 \\#7 & #8 & #9 \end{array} \right)}
\begin{document}

\title{Background}
\maketitle

\ignore{1.  Describe 3-manifold decompositions.
  A.  Prime decomposition theorem -- see Allen Hatcher's notes for statement and references.

%http://www.math.cornell.edu/~hatcher/3M/3M.pdf
}
\ignore{@Misc{•,
OPTkey = {•},
OPTauthor = {Allen Hatcher},
OPTtitle = {Notes on Basic 3-Manifold Topology},
OPThowpublished = {•},
OPTmonth = {•},
OPTyear = {•},
OPTnote = {•},
OPTannote = {•}
}}

The prime decomposition theorem states that a compact, connected, orientable 3-manifold can be decomposed into a connected sum of prime manifolds \cite{Kneser}, and that this is unique up to insertion or deletion of copies of $S^3$ \cite{Milnor}. The connected sum of two spaces is made by removing a ball from each and identifying their bounding spheres.
%What are the prime manifolds? What about manifolds that are not compact or are not orientable?
  
%  B.  Torus decomposition theorem -- see Allen Hatcher's notes for statement and references.

The torus decomposition theorem states that there is a minimal collection of disjointly embedded incompressible tori such that cutting along the edge of each yields components that are each either atoroidal or Seifert-fibered. It also states that this collection is unique up to isomorphism.

%Wikipedia has four references. Which do I use?

%  C.  Geometrization theorem.  (Wikipedia probably has statement, references?)

The geometrization theorem states that any 3-manifold can be decomposed canonically into submanifolds which each have one of the following eight geometries: $S^3, \mathbb{E}^3, \mathbb{H}^3, S^2 \times \mathbb{R}, \mathbb{H}^2 \times \mathbb{R}, \tilde{SL}(2,\mathbb{R}),$ Nil geometry, and Sol geometry. This is done by decomposing along the spheres given by the prime decomposition theorem and tori given by the torus decomposition theorem. \cite{Perelman1} \cite{Perelman2} \cite{Perelman3}

My program will likely never support torus decomposition, so it most likely will never be able to show every manifold.

%What are Nil geometry and Sol geometry? Can my program do this kind of decomposition? How does the decomposition work? \mathbb{E}^3 and \mathbb{H}^3 are homeomorphic. I think they meant that the submanifolds are quotients of those geometries. Can I do all the quotient spaces on \mathbb{H}^3?

%  [D.  Your work:  take some of 8 geometries, glue together via "smooth" prime decomposition.]

My program will allow you to smoothly create a connected sum of several geometries. ``Smoothness'' means that the boundaries that are identified have the same curvatures in each surface. If they are not smoothly identified, then they will appear to have different curvatures from each side. As a result, a geodesic that barely intersects the border can have a very different path than one that barely misses.

%Cone points?

%I also hope to add cone points to my program.


%2.  Visualizing 3-manifolds.
%  A.  Jeff Weeks website -- compact manifolds of constant curvature

There has been previous work in visualizing $S^3$ and $\mathbb{H}^3$ and quotient spaces thereof. In particular, Jeff Weeks has written a program that can view compact quotient spaces of $S^3, \mathbb{E}^3,$ and $\mathbb{H}^3$. \cite{CurvedSpaces}

My program will allow non-compact spaces, and it will allow different spaces to be glued together.

%  B.  Youtube movie

%http://www.spacetimetravel.org/wurmlochflug/wurmlochflug.html


Corvin Zahn made a computer generated video illustrating a wormhole. He used a solution to Einstein's field equations that had previously been found, and simulated the camera moving through this wormhole. He did not go into specifics about the program used. He detailed what kind of manifold was used, but not how he used it. I have emailed him with questions regarding it, but he is yet to respond. \cite{spacetimetravel}

%  C.  Other?  E.g. old references on visualizing hyperbolic geometry?

%3.  Cone manifolds
%  A.  Define.  For example see definition in book "Three-dimensional orbifolds and cone manifolds" by Cooper, Hodgson, Kerckhoff.  Find other references there.
%  [B.  Note they are typically studied for manifolds of constant sectional curvature.  You will look at nonconstant.]

%Is this the same as the Sasakian manifold?

%https://en.wikipedia.org/wiki/Clairaut's_relation Good for solving surfaces of revolution numerically.

\bibliographystyle{plain}
\bibliography{Bibliography}

\end{document}
