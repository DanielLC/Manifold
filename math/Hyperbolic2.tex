\section{3d Hyperbolic Geometry}

\subsection{Finding the direction and distance from one point to another}

We will use the upper half space model: $\mathbb{H}^3 \mapsto \{(x_1,x_2,x_3):x_3 > 0\}$.

Given points $\textbf{x} = (x_1,x_2,x_3)$ and $\textbf{y} = (y_1,y_2,y_3)$, the first step is to find the vertical half-plane that intersects both points. This way, the problem can be reduced to a problem in $\mathbb{H}^2$. Unless both points are on the same vertical line i.e. $x_1 = y_1$ and $x_2 = y_2$, we let the horizontal distance between the two points be $u = \sqrt{(y_1-x_1)^2 + (y_2-x_2)^2}$, then set $\textbf{w} = (w_1,w_2) = (y_1-x_1,y_2-x_2)/u$ to be the unit $2$-vector on the horizontal plane in the direction from $\textbf{x}$ to $\textbf{y}$. We can now work with $(x'_1,x'_2) = (0,x_3)$ and $(y'_1,y'_2) = (u,y_3)$.

Once we have two points on a half-plane $\textbf{x}', \textbf{y}'$, we can use the two-dimensional solution to find the vector $\textbf{v}' = (v'_1,v'_2)$ representing the direction and distance from $\textbf{x}'$ to $\textbf{y}'$.

We now translate the vector back from the half-plane using $\textbf{v} = (v_1,v_2,v_3) = (v'_1w_1,v'_1w_2,v'_2)$.

The distance remains the same as it was in the two-dimensional case.

There is a problem if $x_1 = y_1, x_2 = y_2$ because $\textbf{w}$ is undefined. In this case, $\textbf{v} = (0,0,\ln\frac{y_3}{x_3})$. Alternately, we could use any unit vector for $\textbf{w}$ above, since the horizontal displacement is zero in any direction.

\subsection{Finding the point a given distance in a given direction from another}

Given initial point $\textbf{x} = (x_1,x_2,x_3)$ and vector $\textbf{v} = (v_1,v_2,v_3)$, we first reduce to the $\mathbb{H}^2$ case as before, unless $v_1 = v_2 = 0$. Let $u = \sqrt{v_1^2+v_2^2}, \textbf{w} = (w_1,w_2) = \frac{(v_1,v_2)}{u}$ and solve the two-dimensional case using the point $\textbf{x}' = (x'_1,x'_2) = (0,x_3) \in \mathbb{H}^2$ and vector $\textbf{v}' = (v'_1,v'_2) = (u,v_3) \in \mathbb{R}^2$ to obtain the point $\textbf{y}' \in \mathbb{H}^2$ and the angle of rotation $\theta$.

%Now we just need to map $\textbf{y}'$ back to $\mathbb{H}^3$, which is done via $(y_1,y_2,y_3) = (x'_1,x'_2,y'_2)+y'_1(v_1,v_2,0) = (x'_1+y'_1w_1,x'_2+y'_1w_2,y'_2)$.
%Seems to be wrong.

Now we just need to map $\textbf{y}'$ back to $\mathbb{H}^3$. The first two coordinates are $(x_1,x_2) + y'_1\textbf{w} = (x_1+y'_1w_1,x_2+y'_1w_2)$. The third coordinate is $y'_2$. This gives the full coordinates as $(x_1+y'_1w_1,x_2+y'_1w_2,y'_2)$.

We will also need to find the change in orientation. Our solution in the two-dimensional analogue gave us the angle of rotation, $\theta$. This corresponds to the point of reference being rotated with the rotation matrix $\mat{\cos\theta}{-\sin\theta}{\sin\theta}{\cos\theta}$.

If $\textbf{w}$ happens to be $(1,0)$, then the $\mathbb{H}^2$ we're using is the subspace of $\mathbb{H}^3$ where the second coordinate is constant. In this case, we're just performing that rotation on the first and third coordinates, resulting in the matrix $M = \matc{\cos\theta}{0}{-\sin\theta}{0}{1}{0}{\sin\theta}{0}{\cos\theta}$.

In general, we can rotate the coordinate system so that $\textbf{w} = (1,0)$, perform the rotation $M$, and then rotate it back. This is equivalent to conjugating $M$ with $\matc{w_1}{w_2}{0}{-w_2}{w_1}{0}{0}{0}{1}$. This gives $$\matc{w_1}{-w_2}{0}{w_2}{w_1}{0}{0}{0}{1} \matc{\cos\theta}{0}{-\sin\theta}{0}{1}{0}{\sin\theta}{0}{\cos\theta} \matc{w_1}{w_2}{0}{-w_2}{w_1}{0}{0}{0}{1}.$$

%You can then expand this to a $3 \times 3$ matrix with $\mat{a}{b}{c}{d} \mapsto \matc{a}{0}{b}{0}{1}{0}{c}{0}{d}$ and conjugate it with $\matc{w_1}{w_2}{0}{-w_2}{w_1}{0}{0}{0}{1}$ to get the $3 \times 3$ rotation matrix.

%Again, if the vector is pointed straight up, any value of $\textbf{v}$ will work as long as it's not NaN or something else that behaves oddly when multiplied by zero.

If $v_1 = v_2 = 0$, then $\textbf{y} = (x_1,x_2,x_3 e^{v_3})$ and there is no rotation.