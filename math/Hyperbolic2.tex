\subsection{3d Hyperbolic Geometry}

\subsubsection{Finding the direction and distance from one point to another}

We will use the upper half space model: $\mathbb{H}^3 \mapsto \{(x_1,x_2,x_3):x_3 > 0\}$.

Given points $\textbf{x} = (x_1,x_2,x_3)$ and $\textbf{y} = (y_1,y_2,y_3)$, the first step is to find the vertical plane that intersects both of them. This way, the problem can be reduced to a problem in $\mathbb{H}^2$. Unless $x_1 = y_1$ and $x_2 = y_2$, we let $u = \sqrt{(y_1-x_1)^2 + (y_2-x_2)^2}$, then set $\textbf{v} = (v_1,v_2) = (y_1-x_1,y_2-x_2)/u$. We can now work with $(x'_1,x'_2) = (0,x_3)$ and $(y'_1,y'_2) = (u,y_3)$.

Once we have two points on a plane $\textbf{x}', \textbf{y}'$, we can use the two-dimensional solution to find the vector $\textbf{z}' = (z'_1,z'_2)$ representing the direction and distance from $\textbf{x}'$ to $\textbf{y}'$.

Translating the vector back from the plane is simple. You just use $\textbf{z} = (z_1,z_2,z_3) = (z'_1v_1,z'_1v_2,z'_2)$.

The distance remains the same as it was in the two-dimensional case.

There is a problem if $x_1 = y_1, x_2 = y_2$ because $\textbf{v}$ is undefined. In this case, $\textbf{z} = (0,0,\ln\frac{y_3}{x_3})$ %In this case, any value can be used for $\textbf{v}$, since it passes through every plane. If it's not a unit vector, it will scale the final value on the first two axes, but since it's zero on these axes, it doesn't matter. Just be careful not to use an undefined value.

\subsubsection{Finding the point a given distance in a given direction from another}

Given initial point $(x_1,x_2,x_3)$ and vector $(z_1,z_2,z_3)$, we first reduce to the $\mathbb{H}^2$ case as before, unless $z_1 = z_2 = 0$. Let $u = \sqrt{z_1^2+z_2^2}, (v_1,v_2) = \frac{(z_1,z_2)}{u}$ and solve the two-dimensional case for $\textbf{y}'$ with $\textbf{x}' = (x'_1,x'_2) = (0,x_3)$ and $\textbf{z}' = (z'_1,z'_2) = (u,z_3)$.

Now we just need to map $\textbf{y}'$ back to $\mathbb{H}^3$, which is done via $(y_1,y_2,y_3) = (x'_1,x'_2,y'_2)+y'_1(v_1,v_2,0) = (x'_1+y'_1v_1,x'_2+y'_1v_2,y'_2)$.

We will also need to find the change in orientation.

Our solution in the two-dimensional version gave us an angle which we will call $\theta$. This corresponds to the point of reference being rotated with the rotation matrix $\mat{\cos\theta}{-\sin\theta}{\sin\theta}{\cos\theta}$.

You can then expand this to a $3 \times 3$ matrix with $\mat{a}{b}{c}{d} \mapsto \matc{a}{0}{b}{0}{1}{0}{c}{0}{d}$ and conjugate it with $\matc{v_1}{v_2}{0}{-v_2}{v_1}{0}{0}{0}{1}$ to get the $3 \times 3$ rotation matrix.

This gives $\matc{v_1}{-v_2}{0}{v_2}{v_1}{0}{0}{0}{1} \matc{\cos\theta}{0}{-\sin\theta}{0}{0}{0}{\sin\theta}{0}{\cos\theta} \matc{v_1}{v_2}{0}{-v_2}{v_1}{0}{0}{0}{1}$

%Again, if the vector is pointed straight up, any value of $\textbf{v}$ will work as long as it's not NaN or something else that behaves oddly when multiplied by zero.

If $z_1 = z_2 = 0$, then $\textbf{y} = (x_1,x_2,x_3 e^{z_3})$ and there is no rotation.